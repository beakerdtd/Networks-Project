\documentclass[12pt]{article}
\usepackage{amssymb}
\usepackage{amsmath}
\usepackage{enumerate}
\usepackage{verbatim}
\usepackage{graphicx}
\setlength{\textheight}{8.5in} \setlength{\topmargin}{-.2in}
\setlength{\headheight}{.2in} \setlength{\headsep}{0in}
\date{}
\pagenumbering{gobble}
\begin{document}

\noindent {\bf Note: All definitions, summaries, and inspiration drawn from The \textit {The Economics of Money, Banking, and Financial Markets, Seventh Edition} by Frederic S. Mishkin} \\

\newpage

\begin{center}
\underline{\bf \huge Chapter 1}
\end{center}

\noindent \underline{\bf Why Study Financial Markets?}\\

 \noindent Financial institutions - the systems to be modeled \\

\noindent financial instruments - values or entities exchanged between nodes\\

\noindent node - an agent in the financial system, capable of possessing financial instruments\\

\noindent wealth - a value associated with the financial instruments possessed or attributed to a particular agent\\

\noindent spending - the operation by which the wealth of an agent decreases\\

\noindent borrowing - the operation by which the wealth of an agent increases\\

\noindent financial claim - a condition associated with a financial instrument ensuring that a node is entitled to all or at a least a part of a financial instrument\\

\noindent asset - any financial claim or piece of property that is a store of value\\

\noindent security -  a claim on the borrower's future income that is sold by the borrower to the lender\\

\noindent bond - a debt security that promises to make payments periodically for a specified period of time (time-dependent transfer)\\

\noindent interest rate - a function dictating the cost (or gain) of borrowing  (or lending out) a financial instrument, indicates the amount to be paid to the lender\\

\noindent common stock (stock) - a share of ownership in a corporation, a security that is a claim on the earnings and assets of the corporation\\

\noindent For international markets; way down the road:\\
\noindent foreign exchange market - the financial market in which the conversion between international currencies is determined \\
\noindent conversion - a function for converting one form of currency to another, with the value returned being the \\
\noindent foreign exchange rate - the price of one country's currency in terms of another\\

\noindent \underline{\bf Why Study Banking and Financial Institutions?}\\

\noindent {\bf Structure of the Financial System}\\

\noindent financial intermediaries - nodes that borrow funds from people who have saved and in turn make loans to others\\
\noindent i.e. banks, insurance companies, NBFCs (non-banking financial companies), pension funds (retirement), and investment banks\\

\noindent {\bf Banks and Other Financial Institutions}\\

\noindent banks - financial intermediaries that accept deposits and make loans. Examples include commercial banks, savings and loan associations, mutual savings banks, and credit unions\\

\noindent deposit - a function which subtracts from an agent's wealth and adds to the wealth of a bank, while ensuring a claim on that transfered wealth in the name of the node at hand \\

\noindent loan -a function which subtracts from an agent's (the lender's) wealth and adds to the wealth of another (the borrower), while ensuring a claim on that transfered wealth in the name of the lender\\

\noindent {\bf Financial Innovations}\\

\noindent way down the road: \\
\noindent e-finance - a separate channel of delivering financial services electronically (different edge flavor)\\

\noindent {\bf Money and Business Cycles}\\

\noindent money - anything that is generally accepted in payment for goods or services or the repayment of debts\\

\noindent money supply - the quantity of money in an economy (sum of money over all agents)\\

\noindent aggregate output - the total production of final goods and services in an economy\\
\noindent see Textbooks/Economics.../Chapter1.cpp

\noindent unemployment rate -  a function that returns the percentage of available labor force unemployed\\

\noindent business cycles - the upward and downward movements of aggregate output produced in the economy (a plot of the aggregate output vs time)\\

\noindent recession - any time interval during which the slope a business cycle is negative\\

\noindent monetary theory - the theory that relates changes in the quantity of money to changes in economic activity\\

\noindent {\bf Money and Inflation}\\

\noindent aggregate price level (price level) - the average price of goods and services in an economy\\

\noindent inflation - any time interval during which the slope of the aggregate price level vs time is positive\\

\noindent inflation rate - the rate of change of the price level, usually measured as a percentage change per year\\

\noindent {\bf Money and Interest Rates}\\

\noindent {\bf Conduct of Monetary Policy}\\

\noindent monetary policy - the management of the money supply and interest rates (social philosophy or function), way down the road\\

\noindent central bank -  the government agency that oversees the banking system and is responsible for the amount of money and credit supplied in the economy, a node in the network with the most influence; see Wheeler, Modeling Marx: An Exercise in Futility\\

\noindent Federal Reserve System (the Fed) - the central banking authority responsible for monetary policy in the United States\\

\newpage

\noindent {\bf Fiscal Policy and Monetary Policy}\\

\noindent fiscal policy - (social philosophy) policy that involves decisions about government spending and taxation\\

\noindent budget deficit - the excess of government expenditure over tax revenues, a function that returns an amount in a given currency\\

\noindent budget surplus - the excess of tax revenues over government expenditure, a function that returns an amount in a given currency\\

\noindent gross domestic product (GDP) -  the value of all final goods and services produced in the economy during the course of a year; \\
\noindent see Textbooks/Economics.../Chapter1.cpp\\

\noindent aggregate income - the total income of factors of production (land, labor, capital) in the economy; for now, equivalent to the GDP;\\
\noindent see Textbooks/Economics.../Chapter1.cpp\\

\newpage 

\begin{center}
\large {\bf Chapter 1 Summary}
\end{center}

\noindent 1. Activities in financial markets have direct effects on individuals' wealth, the behavior of businesses, and the efficiency of the economy.\\
Three financial markets deserve particular attention: \\
the bond market (where interest rates are determined),\\
the stock market (which has a major effect on people's wealth and on firms' investment decisions),\\
and the foreign exchange market (because fluctuations in the foreign exchange rate have major consequences for the American economy)\\

\noindent 2. Banks and other financial institutions channel funds from people who might no put them to productive use to people who can do so
and thus play a crucial role in improving the efficiency of the economy.\\

\noindent 3 .Money appears to be a major influence on inflation, business cycles, and interest rates. Because these economic variables are so important to 
the health of the economy, we need to understand how monetary policy is and should be conducted. We also need to study government fiscal policy
because it can be an influential factor in the conduct of monetary policy. \\

\noindent 4. This textbook stresses the economic way of thinking by developing a unifying analytic framework fro the study of money, banking, and financial
markets using a few basic economic principles. This textbook also emphasizes the interaction of theoretical analysis and empirical data.\\

\newpage

\begin{center}
\underline{\bf \huge Chapter 2}
\end{center}

\noindent \underline{\bf Function of Financial Markets} \\

\noindent Types of financial markets: \\
\indent Debt (bonds) and Equity (stock) Markets\\
\indent Primary and Secondary Markets\\
\indent Exchanges and Over-the-Counter Markets\\
\indent Money and Capital Markets\\

\noindent direct finance - the system in which borrowers borrow funds directly from lenders in financial markets by selling them securities \\

\noindent liabilities - IOUs or debts\\

\noindent \underline{\bf Structure of Financial Markets}\\

\noindent {\bf Debt and Equity Markets}\\

\noindent maturity - time to the expiration date (maturity date) of a debt instrument\\

\noindent short-term debt instrument - having a maturity of a year or less\\

\noindent long-term debt instrument - having a maturity of over ten years\\

\noindent intermediate-term debt instrument - having a maturity between one and ten years\\

\noindent equities - claims to share in the net income and assets of a corporation (such as common stocks)\\

\noindent dividends - periodic payments made by equities to shareholders\\

\newpage

\noindent {\bf Primary $\&$ Secondary Markets}\\

\noindent primary market - a financial market in which new issues of a security are sold to initial buyers\\

\noindent secondary market - a financial market in which securities that have been previously issued (and are thus second-hand) can be resold\\
e.g. New York and American Stock Exchange and NASDAQ, bond markets, foreign exchange markets, futures markets, and options markets\\

\noindent investment banks - firms that assist in the initial sale of securities in the primary market\\

\noindent underwriting - guaranteeing prices on securities to corporations and then selling the securities to the public\\

\noindent brokers - agents for investors; they match buyers with sellers\\

\noindent dealers - people who link buyers with sellers by buying and selling securities at stated prices\\

\noindent liquid - easily converted to cash\\

\noindent {\bf Exchanges and Over-the-Counter Markets}\\

\noindent exchanges - secondary markets in which buyers and sellers of securities (or their agents or brokers) meet in one central location to conduct transactions\\
e.g. NYSE, American Stock Exchange, Chicago Board of Trade\\

\noindent over-the-counter (OTC) market == a secondary market in which dealers at different locations who have an inventory of securities stand ready to buy and sell securities "over the counter" to anyone who comes to them and is willing to accept their prices\\
e.g. U.S. government bond market\\

\noindent {\bf Money and Capital Markets}\\

\noindent money market - a financial market in which only short-term debt instruments are traded\\

\noindent capital market - a financial market in which longer-term debt and equity instruments are traded\\

\noindent \underline{\bf Internationalization of Financial Markets}\\

\noindent (way down the road) \\

\noindent {\bf International Bond Market, Eurobonds, $\&$ Eurocurrencies}\\

\noindent foreign bond - bonds sold in a foreign country and denominated in that country's currency\\

\noindent Eurobonds - bonds denominated in a currency other than that of the country in which they are sold\\

\noindent Eurocurrencies - a variant of the eurobond, which are foreign currencies deposited in banks outside the home country\\

\noindent Eurodollar - US. dollars that are deposited in foreign banks outside the U.S. or in foreign branches of U.S. banks\\

\noindent {\bf World Stock Markets}\\

\noindent (way down the road) \\

\noindent \underline{\bf Function of Financial Intermediaries}\\

\noindent indirect finance - the system in which funds are reallocated through the use of financial intermediaries\\

\noindent financial intermediation - the process of indirect finance whereby financial intermediaries link lender-savers with borrower-spenders\\

\noindent {\bf Transaction Costs}\\

\noindent transaction costs - the time and money spent trying to exchange financial assets, goods, or services\\

\noindent economies of scale - the reduction in transaction costs per dollar of transaction as the size (scale) of transactions increases\\

\noindent liquidity services - services that make it easier for customers to conduct transactions\\

\newpage

\noindent {\bf Risk Sharing}\\

\noindent risk - the degree of uncertainty associated with the return on an asset\\

\noindent risk sharing - the process of creating and selling assets with risk characteristics that people are comfortable with and then using the funds 
they acquire by selling these assets to purchase other assets that may have far more risk\\

\noindent asset transformation - the process of turning risky assets into safer assets for investors by creating and selling assets with risk characteristics 
that people are comfortable with and then using the funds they acquire by selling these assets to purchase other assets that may have far more risk\\

\noindent diversification - investing in a collection (portfolio) of assets whose returns do not always move together, with the result that overall risk
is lower than for individual assets\\

\noindent portfolio - a collection of assets\\

\noindent {\bf Asymmetric Information: Adverse Selection and Moral Hazard}\\

\noindent asymmetric information - the unequal knowledge that each party to a transaction has about the other party\\

\noindent adverse selection - the problem created by asymmetric information before a transaction occurs: the people who are the most undesirable from the 
other party's perspective are the ones who are most likely to want to engage in the financial transaction\\

\noindent moral hazard - the risk that one party to a transaction will engage in behavior that is undesirable from the other party's perspective\\

\newpage

\noindent \underline{\bf Financial Intermediaries}\\

\noindent thrift institutions (thrifts) - savings and loan associations, mutual savings banks, and credit unions\\

\noindent Types of Financial Intermediaries:\\

\noindent Depository Institutions\\
\indent Commercial banks\\
\indent Savings and Loans Associations (S$\&$Ls)\\
\indent Mutual Savings Banks\\
\indent Credit Unions\\

\noindent Contractual  Savings Institutions\\
\indent Life Insurance Companies\\
\indent Fire and Casualty Insurance Companies\\
\indent Pension Funds and Government Retirement Funds\\

\noindent Investment  Intermediaries\\
\indent Finance Companies\\
\indent Mutual Funds\\
\indent Money Market Mutual Funds\\

\noindent \underline{\bf Regulation of the Financial System}\\

\noindent {\bf Increasing Information Available to Investors}\\

\noindent {\bf Ensuring the Soundness of Financial Intermediaries}\\

\noindent financial panic - the widespread collapse of financial markets and intermediaries in an economy\\

\noindent Six types of government implemented regulations:\\
	
\indent Restrictions on Entry\\
\indent Disclosure\\
\indent Restrictions on Assets and Activities\\
\indent Deposit Insurance\\
\indent Limits on Competition\\
\indent Restrictions on Interest Rates\\

\newpage
\begin{center}
\large {\bf Chapter 2 Summary}
\end{center}

\noindent 1. The basic function of financial markets is to channel funds from savers who have an excess of funds to spenders who have a shortage of funds.
Financial markets can do this either through direct finance, in which borrowers borrow funds directly from lenders by selling them securities,
or through indirect finance, which involves a financial intermediary that stands between the lender-savers and the borrower-spenders and helps 
transfer funds from one to the other. This channeling of funds improves the economic welfare of everyone in the society because it allows funds
to move from people who have no productive investment opportunities to those who have such opportunities, thereby contributing to increased efficiency
in the economy. In addition, channeling of funds directly benefits consumers by allowing them to make purchases when they need them most.\\

\noindent 2. Financial markets can be classified as debt and equity markets, primary and secondary markets, exchanges and over-the-counter markets, 
and money and capital markets.\\

\noindent 3. An important trend in recent years is the growing internationalization of financial markets. Eurobonds, which are denominated in a currency
other than that of the country in which they are sold, are now the dominant security in the international bond market and have surpassed 
U.S. corporate bonds as a source of new funds. Eurodollars, which are U.S. dollars deposited in foreign banks, are an important source of funds 
for American banks.\\

\noindent 4. Financial intermediaries are financial institutions that acquire funds by issuing liabilities and in turn use those funds to acquire assets by 
purchasing securities or making loans. Financial intermediaries play an important role in the financial system because they reduce transaction costs, 
allow risk sharing, and solve the problems created by adverse selection and moral hazard. As a result, financial intermediaries allow small savers and 
borrowers to benefit from the existence of financial markets, thereby increasing the efficiency of the economy.\\

\noindent 5. The principal financial intermediaries fall into three categories: banks, contractual savings institutions, and investment intermediaries.\\

\noindent 6. The government regulates financial markets and financial intermediaries for two main reasons: to increase the information available to investors 
and to ensure the soundness of the financial system. Regulations include requiring disclosure of information to the public, restrictions on 
who can set up a financial intermediary, restrictions on what assets financial intermediaries can hold, the provision of deposit insurance, 
reserve requirements, and the setting of maximum interest rates that can be paid on checking accounts and savings deposits.\\

\newpage

\begin{center}
\underline{\bf \huge Chapter 3}
\end{center}

\noindent \underline{\bf Meaning of Money}\\

\noindent Money $\equiv$ money supply - anything that is generally accepted in payment for goods or services or in the repayment of debts, distinct from income and wealth. \\

\noindent currency - Paper money (such as dollar bills) and coins (measured in units of value)\\

\noindent wealth - All resources owned by an individual, including all assets (measured in units of value\\

\noindent income - The of earnings per unit time, represented analytically as $\vec i$ (measured in units of value per unit time\\

\noindent \underline{\bf Functions of Money}\\

\noindent {\bf Medium of Exchange}\\

\noindent medium of exchange - Anything that is used to pay for goods and services\\

\noindent barter economy - An economy without money\\

\noindent transaction cost - The time spent trying to exchange goods or services (measured in units of time)\\

\noindent For a commodity to function effectively as money, it has to meet several criteria: \\

\indent 1. It must be easily standardized, making it simple to ascertain its value

\indent 2. It must be widely accepted

\indent 3. It must be divisible, so that it is easy to "make change"

\indent 4. It must be easy to carry

\indent 5. It must not deteriorate quickly

\noindent {\bf Unit of Account}\\

\noindent unit of account - Anything used to measure value in an economy\\

\noindent Aside: The function determining the number of prices needed for N goods is N choose 2.

\noindent {\bf Store of Value}\\

\noindent store of value - A repository of purchasing power over time (not in units of value/time\\

\noindent liquidity - The relative ease and speed with which an asset can be converted to cash (a function returning a value perhaps?)\\

\noindent hyperinflation - An extreme inflation in which the inflation rate exceeds 50$\%$ per month\\

\noindent \underline{\bf Evolution of the Payments System}\\

\noindent payments system - the method of conducting transactions in the economy\\

\noindent {\bf Commodity Money}\\

\noindent commodity money - money made up of precious metals or another valuable commodity\\

\noindent {\bf Fiat Money}\\

\noindent fiat money - paper currency decreed by a government as legal tender but not convertible into coins or precious metal\\

\noindent {\bf Checks}\\

\noindent {\bf Electronic Payment}\\

\noindent {\bf E-Money}\\

\noindent electronic money (e-money) - money that exists only in electronic form and substitutes for cash as well\\

\noindent smart card - a stored-value card that contains a computer chip that lets it be loaded with digital cash from hte owner's bank account whenever needed\\

\noindent e-cash - electronic money that is used on the Internet to purchase goods or services\\

\newpage

\noindent \underline{\bf Measuring Money}\\

\noindent {\bf The Federal Reserve's Monetary Aggregate}\\

\noindent monetary aggregates - The various measures of the money supply used by the Federal Reserve System (M1, M2, and M3)\\

\noindent M1 - A measure of money that includes currency, traveler's checks, and checkable deposits, such as NOW (negotiated order of withdrawal) accounts and ATS (automatic transfer from savings) accounts\\

\noindent M2 - A measure of money that adds to M1: money market deposit accounts, money market mutual fund shares, small-denomination time deposits, savings deposits, overnight repurchase agreements, and overnight Eurodollars \\

\noindent M3 - A measure of money that adds to M2: large-denomination time deposits, long-term repurchase agreements, and institutional money market fund shares\\

\noindent \underline{\bf How Reliable Are the Money Data?}\\

\noindent "We probably should not pay much attention to short-run movements in the money supply numbers, but should be concerned only with longer-run movements. \\

\newpage
\begin{center}
\large {\bf Chapter 3 Summary}
\end{center}

\noindent 1. To economists, the word \textit{money} has a different meaning from \textit{income} and \textit{wealth}. Money is anything that is generally accepted as payment for goods and services.\\

\noindent 2. Money serves three primary functions: as a medium of exchange, as a unit of account, and as a store of value. Money as a medium of exchange avoids the problem of double coincidence of wants that arises in a barter economy by lowering transaction costs and encouraging specialization and the division of labor. Money as a unit of account reduces the number of prices needed in the economy, which also reduces transaction costs. Money also functions as a store of value, but performs this role poorly if it is rapidly losing value due to inflation.\\

\noindent 3. The payments system has evolved over time. Until several hundred years ago, the payments system in all but the most primitive societies was based primarily on precious metals. The introduction of paper currency lowered the cost of transporting money. The next major advance was the introduction of checks, which lowered transaction costs still further. We are currently moving toward and electronic payments system in which paper is eliminated and all transactions are handled by computers. Despite the potential efficiency of such a system, obstacles are slowing the movement to the checkless society and the development of new forms of electronic money. \\

\noindent 4. The Federal Reserve System has defined three different measures of the money supply - M1, M2, and M3. These measures are not equivalent and do not always move together, so they cannot be used interchangeably by policymakers. Obtaining the precise, correct measure of money does seem to matter and has implications for the conduct of monetary policy.  \\

\noindent 5. Another problem in the measurement of money is that the data are not always as accurate as we would like. Substantial revisions in the data do occur; they indicate that initially released money data are not a reliable guide to short-run (say, month-to-month) movements in the money supply, although they are more reliable over longer periods of time, such as a year.\\

\newpage


\end{document}