\documentclass[12pt]{article}
\usepackage{amssymb}
\usepackage{amsmath}
\usepackage{enumerate}
\usepackage{graphicx}
\setlength{\textheight}{8.5in} \setlength{\topmargin}{-.2in}
\setlength{\headheight}{.2in} \setlength{\headsep}{0in}
\date{}
\pagenumbering{gobble}
\begin{document}

\noindent {\bf Note: All definitions, summaries, and inspiration drawn from The \textit {The Economics of Money, Banking, and Financial Markets, Seventh Edition} by Frederic S. Mishkin} \\

 \noindent Financial institutions - the systems to be modeled \\

\noindent financial instruments - values or entities exchanged between nodes\\

\noindent node - an agent in the financial system, capable of possessing financial instruments\\

\noindent wealth - a value associated with the financial instruments possessed or attributed to a particular agent\\

\noindent spending - the operation by which the wealth of an agent decreases\\

\noindent borrowing - the operation by which the wealth of an agent increases\\

\noindent financial claim - a condition associated with a financial instrument ensuring that a node is entitled to all or at a least a part of a financial instrument\\

\noindent asset - any financial claim or piece of property that is a store of value\\

\noindent security -  a claim on the borrower's future income that is sold by the borrower to the lender\\

\noindent bond - a debt security that promises to make payments periodically for a specified period of time (time-dependent transfer)\\

\noindent interest rate - a function dictating the cost (or gain) of borrowing  (or lending out) a financial instrument, indicates the amount to be paid to the lender\\

\noindent common stock (stock) - a share of ownership in a corporation, a security that is a claim on the earnings and assets of the corporation\\

\noindent For international markets; way down the road:\\
\noindent foreign exchange market - the financial market in which the conversion between international currencies is determined \\
\noindent conversion - a function for converting one form of currency to another, with the value returned being the \\
\noindent foreign exchange rate - the price of one country's currency in terms of another\\

\noindent financial intermediaries - nodes that borrow funds from people who have saved and in turn make loans to others\\
\noindent i.e. banks, insurance companies, NBFCs (non-banking financial companies), pension funds (retirement), and investment banks\\

\noindent banks - financial intermediaries that accept deposits and make loans. Examples include commercial banks, savings and loan associations, mutual savings banks, and credit unions\\

\noindent deposit - a function which subtracts from an agent's wealth and adds to the wealth of a bank, while ensuring a claim on that transfered wealth in the name of the node at hand \\

\noindent loan -a function which subtracts from an agent's (the lender's) wealth and adds to the wealth of another (the borrower), while ensuring a claim on that transfered wealth in the name of the lender\\

\noindent way down the road: \\
\noindent e-finance - a separate channel of delivering financial services electronically (different edge flavor)\\

\noindent money - anything that is generally accepted in payment for goods or services or the repayment of debts\\

\noindent money supply - the quantity of money in an economy (sum of money over all agents)\\

\noindent aggregate output - the total production of final goods and services in an economy\\
\noindent see Textbooks/Economics.../Chapter1.cpp

\noindent unemployment rate -  a function that returns the percentage of available labor force unemployed\\

\noindent business cycles - the upward and downward movements of aggregate output produced in the economy (a plot of the aggregate output vs time)\\

\noindent recession - any time interval during which the slope a business cycle is negative\\

\noindent monetary theory - the theory that relates changes in the quantity of money to changes in economic activity\\

\noindent aggregate price level (price level) - the average price of goods and services in an economy\\

\noindent inflation - any time interval during which the slope of the aggregate price level vs time is positive\\

\noindent inflation rate - the rate of change of the price level, usually measured as a percentage change per year\\

\noindent monetary policy - the management of the money supply and interest rates (social philosophy or function), way down the road\\

\noindent central bank -  the government agency that oversees the banking system and is responsible for the amount of money and credit supplied in the economy, a node in the network with the most influence; see Wheeler, Modeling Marx: An Exercise in Futility\\

\noindent fiscal policy - (social philosophy) policy that involves decisions about government spending and taxation\\

\noindent budget deficit - the excess of government expenditure over tax revenues, a function that returns an amount in a given currency\\

\noindent budget surplus - the excess of tax revenues over government expenditure, a function that returns an amount in a given currency\\

\noindent gross domestic product (GDP) -  the value of all final goods and services produced in the economy during the course of a year; \\
\noindent see Textbooks/Economics.../Chapter1.cpp\\

\noindent aggregate income - the total income of factors of production (land, labor, capital) in the economy; for now, equivalent to the GDP;\\
\noindent see Textbooks/Economics.../Chapter1.cpp\\

\newpage

Chapter 1 Summary\\


\noindent 1. Activities in financial markets have direct effects on individuals' wealth, the behavior of businesses, and the efficiency of the economy.\\
Three financial markets deserve particular attention: \\
the bond market (where interest rates are determined),\\
the stock market (which has a major effect on people's wealth and on firms' investment decisions),\\
and the foreign exchange market (because fluctuations in the foreign exchange rate have major consequences for the American economy)\\

\noindent 2. Banks and other financial institutions channel funds from people who might no put them to productive use to people who can do so
and thus play a crucial role in improving the efficiency of the economy.\\

\noindent 3 .Money appears to be a major influence on inflation, business cycles, and interest rates. Because these economic variables are so important to 
the health of the economy, we need to understand how monetary policy is and should be conducted. We also need to study government fiscal policy
because it can be an influential factor in the conduct of monetary policy. \\

\noindent 4. This textbook stresses the economic way of thinking by developing a unifying analytic framework fro the study of money, banking, and financial
markets using a few basic economic principles. This textbook also emphasizes the interaction of theoretical analysis and empirical data.\\


\end{document}